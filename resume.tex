\documentclass[letterpaper,10pt]{article}

\usepackage{latexsym}
\usepackage[empty]{fullpage}
\usepackage{titlesec}
\usepackage{marvosym}
\usepackage[usenames,dvipsnames]{color}
\usepackage{verbatim}
\usepackage{enumitem}
\usepackage[hidelinks]{hyperref}
\usepackage{fancyhdr}
\usepackage[english]{babel}
\usepackage{tabularx}
\usepackage{multicol}
\usepackage{fontawsome}
\input{glyphtounicode}

\usepackage[default]{sourcesanspro}
\usepackage[T1]{fontenc}

\pagestyle{fancy}
\fancyhf{} 
\fancyfoot{}
\renewcommand{\headrulewidth}{0pt}
\renewcommand{\footrulewidth}{0pt}


\addtolength{\oddsidemargin}{-0.5in}
\addtolength{\evensidemargin}{-0.5in}
\addtolength{\textwidth}{1in}
\addtolength{\topmargin}{-.5in}
\addtolength{\textheight}{1.0in}

\urlstyle{same}

\raggedbottom
\raggedright
\setlength{\tabcolsep}{0in}

\titleformat{\section}{
  \vspace{10pt}\centering
}{}{0em}{}[\vspace{2pt} \color{black}\titlerule\vspace{2pt}]


\pdfgentounicode=1

\newcommand{\resumeItem}[1]{
  \item\normalsize{
    {#1 \vspace{-2pt}}
  }
}

\newcommand{\resumeSubheading}[4]{
  \vspace{-2pt}\item
    \begin{tabular*}{0.97\textwidth}[t]{l@{\extracolsep{\fill}}r}
      \textbf{#1} & #2 \\
      \textit{\small#3} & \textit{\small #4} \\
    \end{tabular*}\vspace{-7pt}
}

\newcommand{\resumeSubSubheading}[2]{
    \item
    \begin{tabular*}{0.97\textwidth}{l@{\extracolsep{\fill}}r}
      \textit{\small#1} & \textit{\small #2} \\
    \end{tabular*}\vspace{-7pt}
}

\newcommand{\resumeProjectHeading}[2]{
    \item
    \begin{tabular*}{0.97\textwidth}{l@{\extracolsep{\fill}}r}
      \small#1 & #2 \\
    \end{tabular*}\vspace{-7pt}
}

\newcommand{\resumeSubItem}[1]{\resumeItem{#1}\vspace{-4pt}}

\renewcommand\labelitemii{$\vcenter{\hbox{\tiny$\bullet$}}$}

\newcommand{\resumeSubHeadingListStart}{\begin{itemize}[leftmargin=0.15in, label={}]}
\newcommand{\resumeSubHeadingListEnd}{\end{itemize}}
\newcommand{\resumeItemListStart}{\begin{itemize}}
\newcommand{\resumeItemListEnd}{\end{itemize}\vspace{-5pt}}

\begin{document}



\begin{center}
    {\fontsize{40pt}{36pt}\selectfont Manan Jain} \\ \vspace{2pt}
    
\end{center}

\vspace{10pt}

% %-----------Contact Info-----------
% \section{\Large Contact Info}
\begin{center}
        \setlength{\tabcolsep}{18pt}
    \begin{tabular}{c | c | c | c }
         {\underline{\faEnvelopeO \hspace{1pt} mananj146@gmail.com}} & {\href{https://github.com/manan3172003}{\underline{\faGithub \hspace{1pt} GitHub}}} & {\href{https://www.linkedin.com/in/manan-jain-253486224/}{\underline{\faLinkedinSquare \hspace{1pt} LinkedIn}}} & {\faPhone \hspace{1pt} {+1 (587) 938 5436}}
    \end{tabular}{}
\end{center}
\vspace{5pt}

%-----------PROGRAMMING SKILLS-----------
\section{\LARGE Technical Skills}
 \begin{itemize}[leftmargin=0.15in, label={}]
    \small{\item{
     \textbf{\normalsize{Programming Languages, DBMS, other tools:}}
        {\resumeItemListStart
        \resumeItem{Proficient: C/C++, Python, RISC-V Assembly, Android Studio, Java, Git, GitHub, WSL}
        \vspace{3pt}
        \resumeItem{Intermediate: Arduino, MySQL, JavaScript, HTML/CSS, Kotlin}
        \vspace{3pt}
        \resumeItem{Familiar: xml, React, Express}
    \resumeItemListEnd}
    \vspace{5pt}
     \textbf{\normalsize{Languages}}{\normalsize{: Hindi (native), English (fluent), French (Beginner)}}
     
    }}
 \end{itemize}


%-----------EDUCATION-----------
\vspace{-2pt}
\section{\LARGE Education}
  \resumeSubHeadingListStart
      \resumeSubheading
      {\large \faMortarBoard \hspace{1pt} University of Alberta}{\large Sept. 2021 -- Present}
      {\normalsize{BSc Honors in Computer Science}}{\normalsize{Edmonton, AB}}
      \resumeSubheading
      {}{}
      {\normalsize{Relevant Coursework:}}{}
      \vspace{10pt}
  \resumeSubHeadingListEnd
        \resumeItemListStart
        \resumeItem{Introduction to Software Engineering (Unified Modeling Language, Software architecture, frameworks etc.)}
        \resumeItem{Introduction to File and Database Management (Entity-relation model, storage architecture etc.)}
        \resumeItem{Basics of Machine Learning}
        \resumeItem{Algorithms 1 (Sorting Algorithms, Graph Algorithms, Dynamic Programming etc)}
        \resumeItem{Computer Organization and Architecture 1 (Assembly Level programming, Instruction Set Architecture, pipelining, virtual memory etc.)}
        \resumeItem{Tangible Computing 1 and 2 (major focus on object oriented programming and complex algorithms that include graphing, caching, memoization)}
    \resumeItemListEnd


%-----------EXPERIENCE-----------
\section{\LARGE Projects}
 \vspace{5pt}
  \resumeSubHeadingListStart
    \resumeSubheading
      {\href{https://github.com/manan3172003/votebased-chatbox}{\underline{\faCode \hspace{1pt} Vote-Based Chatbox (JavaScript, MongoDB)} }}{}
      {}{}
      \resumeItemListStart
        \resumeItem{built a software with code-based chatrooms where messages are sent and then upvoted or downvoted, with the highest-rated message appearing at the top.}
        \vspace{1pt}
        \resumeItem{uses the JavaScript React framework for the front end and the JavaScript Express framework and MongoDB for the back end.}
        \vspace{1pt}
        \resumeItem{can be used by instructors teaching online classes to give priority to responding to the most frequently asked questions.}
    \resumeItemListEnd
    
    \vspace{10pt}

    
    \resumeSubheading
      {\href{https://github.com/manan3172003/sonar-arduino}{\underline{\faCode \hspace{1pt} Arduino Radar (Java, C++)} }}{}
      {}{}
      \resumeItemListStart
        \resumeItem{built a device that measures the separation between an object and itself}
        \resumeItem{uses an arduino UNO chip, breadboard, servo motor and an HC-05 ultrasonic sensor}
        \resumeItem{provides a visual of the location of the object with respect to itself}
    \resumeItemListEnd
     
    \vspace{10pt}
    
    \resumeSubheading
      {\href{https://github.com/manan3172003/Snake-Game}{\underline{\faCode \hspace{1pt} Snake-Game (RISC-V assembly)} }}{}
      {}{}
      \resumeItemListStart
        \resumeItem{built a snake game with 3 different difficulty levels}
        \resumeItem{uses RISC-V assembly language and user and timer interrupts.}
        \resumeItem{for a better understanding of interrupts and exception handlers}
    \resumeItemListEnd
    
    \vspace{15pt}
  \resumeSubHeadingListEnd

\end{document}
